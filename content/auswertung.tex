\section{Auswertung}
\label{sec:Auswertung}
Zur Bestimmung der Dichte der beiden Kugeln sind die Masse und das Volumen beider Kugeln benötigt. Diese sind in \autoref{tab:Kugeln}
zu sehen. Die Dichte bestimmt sich dann über die Relation
\begin{equation}
\rho = \frac{m}{V}.
\end{equation}
Somit ergeben sich die Dichten der beiden Kugeln
\\ \\
\centerline{$\rho_\text{kl} = 2543.58 \frac{\symup{kg}}{\symup{m^3}}$,}
\centerline{$\rho_\text{gr} = 2567.56 \frac{\symup{kg}}{\symup{m^3}}$.}
\\ \\

Des Weiteren ist die Dichte von destilliertem Wasser, also die in der Versuchsreihe genutzte Flüssigkeit, bei Raumtemperatur (20°C) \cite{science}
\\ \\
\centerline{$\rho_\text{Fl} = 998.20 \frac{\symup{kg}}{\symup{m^3}}$}.
\\ \\

Zur Bestimmung der Proportionalitätskonstanten $K$ wird zunächst über die Fallzeiten, aufgelistet in \autoref{tab:fallzeit}, der beiden Kugeln, unter Verwendung der Gleichungen \eqref{eqn:mittelwert} und \eqref{eqn:FehlerMittelwert},
gemittelt, so dass  sich folgende gemittelte Zeiten ergeben:
\\ \\
\centerline{$t_\text{kl} = (11.67 \pm 0.04) \symup{s}$}

\centerline{$t_\text{gr} = (76.59 \pm 0.08) \symup{s}$}.
\\ \\

Zudem ist zur Berechnung ebenfalls noch die Viskosität $\eta$ benötigt. Zu dessen Bestimmung werden die aus der Anleitung \cite{V207} gegebenen Angaben zur kleinen Kugel verwendet:
\\ \\
\centerline{$K_\text{kl} = 0.07640 \symup{\frac{Pa \, cm^3}{g}}$}
\centerline{$m_\text{kl} = 4.4531 \symup{g}$}
und daher
\\ \\
\centerline{$\rho_\text{kl} = 2368.67 \frac{\symup{kg}}{\symup{m^3}}$.}
\\ \\

Unter Verwendung von Gleichung \eqref{eqn:viskositaet} und den Werten für die kleine Kugel, ergibt sich Viskosität zu
\\ \\
\centerline{$\eta = (1.222 \pm 0.004) \cdot 10^{-3} Pa \, s$,}
\\ \\
wobei sich der Fehler mittels Gleichung \eqref{eqn:fehlerfortpflanzung} erechnet und somit
\begin{equation}
\Delta \eta = \sqrt{(K(\rho_\text{kl} - \rho_\text{Fl}))^2 \cdot (\Delta t_\text{kl})^2}
\end{equation}
ist.
Mithilfe der Viskosität lässt sich nun die Proportionalitätskonstante für die große Kugel erechnen, in dem Gleichung \eqref{eqn:viskositaet}
umgestellt wird.
Somit gilt:
\begin{equation}
K_\text{gr} = \frac{\eta}{(\rho_\text{gr} - \rho_\text{Fl})t_\text{gr}} = (1.016 \pm 0.004) \cdot 10^{-8} \frac{\symup{Pa \, m^3}{kg}},
\end{equation}
wobei sich auch hier der Fehler wieder mittels Gleichung \eqref{eqn:fehlerfortpflanzung} zu 
\\ \\
\centerline{$\Delta K_\text{gr} = \sqrt{(\frac{1}{(\rho_\text{gr} - \rho_\text{Fl})t_\text{gr}})^2 \cdot (\Delta \eta)^2 + ( \frac{\eta}{(\rho_\text{gr} - \rho_\text{Fl}))^2 \cdot (\Delta t_\text{gr})^2 } } $}
\\ \\
erechnet.
\begin{table}[!htp]
\centering
\caption{Geometrische Daten der beiden Kugeln}
\label{tab:Kugeln}
\begin{tabular}{c c c c c c}
\toprule
\multicolumn{3}{c}{Kleine Kugel} & \multicolumn{3}{c}{Große Kugel} \\
\midrule
{$m$ / g} & {$r$ / cm} & {$V / m^3$ } & {$m$ / g} & {$r$ / cm} & {$V / m^3$ } \\
4.77 & 0.765 & $1.88 \cdot 10^{-6}$ & 4.91 & 0.770 & $1.91 \cdot 10^{-6}$ \\
\bottomrule
\end{tabular}
\end{table}
\begin{table}[!htp]
\centering
\caption{Fallzeiten beider Kugeln bei 20°C}
\label{tab:fallzeit}
\begin{tabular}{c c}
\toprule
{{$t_k$ / s}} & {{$t_g$ / s}} \\
\midrule
11.83 & 76.91 \\
11.69 & 76.07 \\
11.61 & 76.59 \\
11.69 & 76.43 \\
11.63 & 76.49 \\
11.57 & 76.92 \\
11.49 & 76.57 \\
11.84 & 76.69 \\
11.50 & 76.62 \\
11.82 & 76.59 \\
\bottomrule
\end{tabular}
\end{table} 

Zur Berechnung der Temperaturabhängigkeit der Viskosität wird die zuvor durchgeführte Rechnung zur Bestimmung der Viskosität für
alle Messungen wiederholt, hierzu wird die zuvor berechnete Proportionalitätskonstante $K_\text{gr}$ und die Dichte der Kugel $\rho_\text{gr}$ verwendet. Die verwendeten Dichten 
des destillierten Wassers $\rho_\text{Fl}$ enstammen der Literatur \cite{science}.
Die verwendeten und erechneten Werte finden sich in \autoref{tab:visko}.
Zur temperaturabhängigen Bestimmung der Viskosität werden die erechneten Viskositäten logarithmisch gegen den Kehrwert der Temperaturen aufgetragen, wie in \autoref{fig:gerade} zu sehen.
Die Werte hierzu finden sich in \autoref{tab:geradewerte}. Nach Gleichung \eqref{eqn:tempabhaengig} (Andradeschen Gleichung) wird eine lineare Ausgleichsrechung der Form
\\ \\
\centerline{$\ln(\eta) = \ln(A) + B \cdot \frac{1}{T}$,}
\\ \\
wobei $A$ und $B$ Konstanten sind.
Die Konstanten $A$ und $B$ sind damit:
\\ \\
\centerline{$A = (-12.56 \pm 0.08)$}
\centerline{$B = (1722 \pm 26 )$}.
\\ \\
\begin{table}[!htp]
\centering
\caption{Messdaten und erechnete Werte zur Bestimmung der Temperaturabhängigkeit der Viskosität.}
\label{tab:visko}
\begin{tabular}{c c c S[table-format=1.3] @{${}\pm{}$} S[table-format=1.3]}
\toprule
{{$T$ / K}} & {{$t$ / s}} & {{$\rho_\text{Fl} $ / $\frac{\symup{kg}}{\symup{m^3}}$}} & \multicolumn{2}{c}{$\eta$ / $10^{-3} \symup{Pa} \, \symup{s}$} \\
\midrule
295.15 & 76.62 & 997.77 & 1.222  & 0.004 \\
295.15 & 76.59 & 997.77 & 1.221  & 0.004 \\
304.15 & 63.00 & 995.34 & 1.006  & 0.004 \\
304.15 & 62.71 & 995.34 & 1.002  & 0.004 \\
307.15 & 59.14 & 994.37 & 0.945  & 0.003 \\
307.15 & 59.58 & 994.37 & 0.952  & 0.003 \\
312.15 & 54.88 & 992.59 & 0.878  & 0.003 \\
312.15 & 54.91 & 992.59 & 0.878 & 0.003 \\
316.15 & 51.14 & 991.03 & 0.819  & 0.003 \\
316.15 & 50.10 & 991.03 & 0.802 & 0.002 \\
320.15 & 47.96 & 989.36 & 0.769  & 0.002 \\
320.15 & 47.09 & 998.36 & 0.751  & 0.002 \\
324.15 & 44.73 & 987.58 & 0.718  & 0.002\\
324.15 & 44.32 & 987.58 & 0.711  & 0.002\\
329.15 & 40.15 & 985.20 & 0.645  & 0.002\\
329.15 & 40.65 & 985.20 & 0.653  & 0.002 \\
334.15 & 38.52 & 982.15 & 0.620  & 0.002 \\
334.15 & 38.64 & 982.15 & 0.622  & 0.002 \\
\bottomrule
\end{tabular}
\end{table}
\begin{table}[!htp]
\centering
\caption{Zum Erstellen der Ausgleichsgerade verwendete Werte.}
\label{tab:geradewerte}
\begin{tabular}{ S[table-format=1.3] S[table-format=1.3]}
\toprule
{$\ln(\eta)$} & {$\frac{1}{T}$ / $10^{-3} K^{-1}$} \\
\midrule
-6.706 & 3.388 \\
-6.707 & 3.388 \\
-6.901 & 3.287\\
-6.905 & 3.287\\
-6.963 & 3.255\\
-6.956 & 3.255\\
-7.037 & 3.203 \\
-7.036 & 3.203 \\
-7.106 & 3.163\\
-7.127 & 3.163\\
-7.170 & 3.123\\
-7.194 & 3.123\\
-7.238 & 3.084 \\
-7.247 & 3.084 \\
-7.345 & 3.038 \\
-7.332 & 3.038 \\
-7.384 & 2.992\\
-7.381 & 2.992\\
\bottomrule
\end{tabular}
\end{table}
\begin{figure}
  \centering
  \includegraphics{gerade.pdf}
  \caption{Ausgleichsgerade zur Bestimmung der Konstanten der Andradeschen Gleichung.}
  \label{fig:gerade}
\end{figure}
Zur Beurteilung, ob die Strömung in dem Viskosimeter tatsächlich laminar ist, wird die Reynoldszahl nach Gleichung \eqref{eqn:reynold} bestimmt.
Hierbei ist $\rho$ die Dichte der Flüssigkeit, $v= \frac{x}{t}$ die Geschwindigkeit der Kugel mit der Fallstrecke $x=0.1 \symup{m}$ und der Fallzeit $t$, sowie der 
Viskosität $\eta$ und $d = 2\cdot r$ der Durchmesser der Kugel. Somit ergibt sich die Reynoldszahl zu 
\begin{equation}
Re = \frac{\rho_\text{Fl} 2r x}{t \eta}
\end{equation}
und es wird deutlich, dass für die Berechnung eines Maximums der Reynoldszahl der Wert aus \autoref{tab:visko}, welcher die geringste Viskosität und geringste Fallzeit aufweist.
Somit ergibt sich für die Reynoldszahl:
\\ \\
\centerline{$ Re = 63.53 \pm 0.2$.}
\\ \\
Der Fehler der Reynoldszahl erechent sich mittels Gleichung \eqref{eqn:fehlerfortpflanzung} zu
\begin{equation}
\Delta Re = \sqrt{\Bigl(- \frac{\rho v d}{\eta^2}\Bigr)^2 \cdot \eta^2}.
\end{equation}
