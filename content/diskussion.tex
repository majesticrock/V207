\section{Diskussion}
\label{sec:Diskussion}
Der berechnete Proportionalitätsfaktor $K_\text{gr}$ liegt in der selben Größenordnung vor, wie der Faktor $K_\text{kl}$, weshalb 
davon auszugehen ist, dass der erechnete Wert innerhalb der Messgenauigkeit liegt. Des Weiteren ist der Fehler der Proportionalitätskonstante
niedrig (ca. 0.4\% Abweichung), was auf eine präzise Messung schließen lässt, welche unter anderem durch die hohe Anzahl an Messungen
zustande kommt. Dennoch ist davon auszugehen, dass weder der Wert der Konstante $K$ noch die Viskosität $\eta$ genau ist, da es
unmöglich ist, einerseits die Apparatur vollständig von Luftbläschen zu befreien, welche für eine erhöhte Reibung sorgen und so die 
Ergebnisse verfälscht, andererseits ist es die begrenzte menschliche Genauigkeit bei der Messung der Zeit. 
Dies erklärt auch die Abweichung der erechneten Viskosität um ca. 17 \% vom Literaturwert ($\eta = 1.0087 \symup{Pa \, s}$) \cite{viskos},
welche aber dennoch im Bereich der Messgenauigkeit liegt.
Die Luftbläschen waren 
besonders bei der Messung der Temperaturabhängigkeit der Viskosität hinderlich, weshalb der Versuch bereits nach neun Messungen 
abgebrochen werden musste. 
Die Konstanten der Andradeschen Gleichung sind aber aufgrund der vergleichsweisen geringen statistischen Fehler als richtig 
anzusehen. 
Wie ebenfalls zu erwarten, verhält sich die Strömung in der Apparatur tatsächlich laminar, was durch die die Reynoldszahl belegt wird.
Diese liegt nämlich deutlich unter dem kritischen Wert von 2300 \cite{reynold}.
Somit sind auch alle Annahmen, die eine laminare Strömung zugrunde legen, als richtig anzusehen.