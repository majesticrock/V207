\section{Zielsetzung}

Die Apperaturkonstante des verwendeten Viskosimeters mit zwei unterschiedlichen Kugeln wird mithilfe von diesem Versuch bestimmt.
Des Weiteren wird die Reynoldszahl bestimmt, um so zu festzustellen, ob die Flüssigkeit laminares Verhalten hat. 
Die Viskosität der Flüssigkeit wird weiterhin auf Temperaturabhängigkeit untersucht.

\section{Theorie}
\label{sec:Theorie}

Auf einen bewegten Körper wirken in einer Flüssigkeit diverse Kräfte. Es wirkt die aus der Gravitation resultierende Gewichtskraft $\symbf{F}_G$. Diese wirkt senkrecht Richtung Boden.
Antiparallel zu dieser wird die Auftriebskraft $\symbf{F}_A$. Diese verhält sich proportional zu dem Volumen der verdrängten Flüssigkeit.
Des Weiteren wirkt die Reibungskraft $\symbf{F}_R$, welche entgegen der Bewegungsrichtung wirkt.
Ist die Strömung in der Flüssigkeit laminar, das heißt, es bilden sich keine Wirbel und Turbulenzen, so gilt die Reibung nach Stokes:

\begin{equation}
\label{eqn:stokes}
    \symbf{F}_R = 6 \pi \eta v r .
\end{equation}

Dabei ist $\eta$ die Viskosität der Flüssigkeit, $v$ die Geschwindigkeit mit der sich ein Körper relativ zu der Flüssigkeit bewegt und $r$ ist der Radius der hier genutzten Kugel.
Es wird sich hier vorgestellt, dass die Flüssigkeit aus verschiedenen Schichten besteht, die aneinander reiben.
Die Reibungskraft steigt mit der Geschwindigkeit, daher bildet sich ab einer gewissen Geschwindigkeit ein Kräftegleichgewicht, sodass diese konstant bleibt.

Um zu bestimmen, ob eine Flüssigkeit laminare Eigenschaften zeigt, wird die Reynoldszahl $Re$ genutzt. Diese ist durch

\begin{equation}
\label{eqn:reynold}
    Re = \frac{\rho v d}{\eta}
\end{equation}

definiert. Dabei ist $\rho$ die Dichte der Flüssigkeit und $d$ der Durchmesser des Rohres. 
Die in dieser Gleichung ebenfalls vorkommende Viskosität $\eta$ ist stark von der Temperatur abhängig, was die sogenannte Adradesche Gleichung

\begin{equation}
\label{eqn:tempabhaengig}
    \eta(T) = A \cdot \e^\frac{B}{T}
\end{equation}

zeigt. Dabei sind A und B Konstanten, die im Laufe des Versuchs bestimmt werden. 
Ist die resultierende Reynoldszahl kleiner als ein bestimmter Wert, so handelt es sich um eine laminare Strömung und es kann die oben genannte Reibung nach Stokes verwendet werden.
